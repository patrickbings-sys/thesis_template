% Dokumentenklasse -----------------------------------------------------------------------------------------------------

\documentclass[a4paper, 12pt]{scrreprt} % Dokumentenklasse für Berichte im KOMA-Script-Stil

% Einrückung entfernen und Absatzabstand setzen
\setlength{\parindent}{0em} % Entfernt die Einrückung für Absätze
\setlength{\parskip}{1em}   % Setzt einen Abstand von 1em zwischen Absätzen

% Pakete ---------------------------------------------------------------------------------------------------------------
\usepackage[left=3cm, right=2cm,top=2cm, bottom=3cm]{geometry}   % Anpassung der Seitenränder
\usepackage[utf8]{inputenc}   % Ermöglicht die Verwendung von UTF-8 Zeichenkodierung
\usepackage[ngerman]{babel}   % Sprachunterstützung für Deutsch, inklusive Trennregeln und Datum
\usepackage{csquotes} % Hinzufügen des csquotes-Pakets für bessere Handhabung von Zitaten
\usepackage[T1]{fontenc}   % Verwendung von T1-Schriftkodierung für bessere Ausgabequalität
\usepackage{graphicx}   % Einbindung von Grafiken
\usepackage{verbatim}
\usepackage{pmboxdraw}
\graphicspath{{img/}}   % Pfad zu den Grafikdateien
\usepackage{lmodern}   % Verwendung der Latin Modern Schriftart
% \usepackage{mathptmx}  % Verwendung von Times New Roman Schriftart
\usepackage{color}   % Farbunterstützung
\usepackage{xcolor}   % Farbunterstützung
\usepackage[printonlyused, withpage]{acronym}   % Verwaltung von Abkürzungen
\usepackage{pdfpages}    % Einbindung von kompletten PDF-Seiten
\usepackage{float}   % Bessere Kontrolle über die Platzierung von Gleitobjekten (z.B. Bilder, Tabellen)
\usepackage{placeins}
\usepackage{array, ragged2e, tabularx}  % Bessere Kontrolle über Tabellen
\usepackage{makecell}
\usepackage{lipsum}
\usepackage[font=small]{caption}   % Anpassung von Bild- und Tabellenunterschriften
\usepackage{subcaption}   % Unterstützung von Unterbeschriftungen für Bilder und Tabellen
\usepackage[
    pdftitle={Hier Titel eintragen}, % Setzt den Titel des PDFs
    pdfsubject={Hier Thema eintragen z.B. Masterarbeit},   % Setzt das Thema des PDFs (hier leer gelassen)
    pdfauthor={Name des Authors}, % Setzt den Autor des PDFs
    pdfkeywords={Keywords eintragen z.B. EEG etc.},  % Setzt die Schlüsselwörter des PDFs (hier leer gelassen)
    hidelinks        % Verhindert die Einrahmung von Links
]{hyperref}
\renewcommand{\UrlFont}{\rmfamily}  % Schriftart der URLs wie die normale Schriftart
\usepackage{amsmath, amssymb}   % Erweiterte mathematische Umgebungen und Symbole
\usepackage{scrhack}  % Korrekturen für die Verwendung bestimmter Pakete mit KOMA-Script-Klassen

% Anpassungen für automatische Referenzierung --------------------------------------------------------------------------
\addto\extrasngerman{\def\figureautorefname{Abb.}} % ersetzt Abbildung durch Abb. bei der referenzierung
\addto\extrasngerman{\def\sectionautorefname{Kapitel}} % erstetzen durch Kapitel bei der referenzierung
\addto\extrasngerman{\def\subsectionautorefname{Kapitel}} % erstetzen durch Kapitel bei der referenzierung

% Zeilenumbrüche -------------------------------------------------------------------------------------------------------
% \hyphenation{

% }

% Literaturverzeichnis -------------------------------------------------------------------------------------------------
\usepackage[backend=biber, 
			style=ieee,
			doi=true,
			url=false,
			eprint=false]{biblatex} 
\addbibresource{literature/Literaturverzeichnis.bib} % Pfad zur .bib-Datei 

% Formelverzeichnis ----------------------------------------------------------------------------------------------------
\makeatletter
\newcommand{\formelref}[1]{Formel~\textup{\tagform@{\ref{#1}}}}
\makeatother

% Kopf- und Fußzeilenanpassung -----------------------------------------------------------------------------------------
\usepackage[autooneside,headsepline]{scrlayer-scrpage}
\automark[chapter]{chapter}
\clearpairofpagestyles
\ihead{\headmark} 		
\ofoot[\pagemark]{\pagemark}

% Dokumentenbeginn -----------------------------------------------------------------------------------------------------

\begin{document}

	\pagenumbering{Roman}

	% Titelseite 
\thispagestyle{empty} % Seitenstil auf 'empty' setzen

\begin{flushright} % Bild rechtsbündig einfügen
	\includegraphics[width=5cm]{img/Allgemein/logo_uni.jpeg}
\end{flushright}

\hrulefill % Horizontale Linie einfügen
\vspace{0.5cm} % Vertikaler Abstand von 1 cm

\begin{center} % Titel zentriert und in großer Schrift
	% \LARGE{\textsc{Implementierung von FastICA in Python zur 
	\LARGE{Hier könnte Ihr Titel stehen}
\end{center}

\vspace{0.5cm} % Vertikaler Abstand von 1 cm

\begin{center} % Typ der Arbeit zentriert und fett
	\textbf{\Large{Typ der Arbeit, z.B. Masterarbeit}}
\end{center}

\vspace{0.25cm} % Vertikaler Abstand von 0.5 cm

\begin{center} % Text "vorgelegt von" zentriert
	vorgelegt von
\end{center}

\vspace{0.25cm} % Vertikaler Abstand von 0.5 cm

\begin{center} % Name und Matrikelnummer zentriert und fett
	\large{\textbf{Hier Namen eintragen}} \\
	\large{\textbf{Matrikelnummer: 1234567}}
\end{center}

\vspace{1cm} % Vertikaler Abstand von 1 cm

\begin{center} % Universitätsinformationen zentriert
	\large{Übergeordnete Affiliation \\ 
		Fakultät eintragen \\ 
		Abteilung eintragen \\ 
		Fachgebiet eintragen}
\end{center}

\vspace{1cm} % Vertikaler Abstand von 1 cm

\begin{center} % Prüferinformationen in einer Tabelle zentriert
	\begin{tabular}{ll}
		\textbf{Erstprüfer:} & Hier Erstprüfer eintragen \\
		\textbf{Zweitprüfer:} & Hier Zweitprüfer eintragen \\
		\textbf{Betreuer:} & Hier Betreuer eintragen \\
	\end{tabular}
\end{center}

\vspace{1cm} % Vertikaler Abstand von 1 cm

\begin{center} % Abgabedatum zentriert und klein
	\small{Datum der Abgabe: TT.MM.JJJJ}
\end{center}

\clearpage % Seitenumbruch und Zurücksetzen des Seitenstils für die folgenden Seiten
 % Einbindung der Titelseite

	\includepdf[
		pages={1},
		scale=0.95, 
		pagecommand={\thispagestyle{plain}}
	]{img/Allgemein/Aufgabenstellung.pdf} % Einbinden der Aufgabenstellung
	
	% Versicherung an Eides Statt ------------------------------------------------------------------------------------------
\chapter*{Versicherung an Eides Statt}

Ich versichere an Eides Statt durch meine Unterschrift, dass ich die vorstehende
Arbeit selbstständig und ohne fremde Hilfe angefertigt und alle Stellen,
die ich wörtlich oder annähernd wörtlich aus Veröffentlichungen entnommen
habe, als solche kenntlich gemacht habe, mich auch keiner anderen als der
angegebenen Literatur oder sonstiger Hilfsmittel bedient habe.
Ich versichere an Eides Statt, dass ich die vorgenannten Angaben nach bestem
Wissen und Gewissen gemacht habe und dass die Angaben der Wahrheit entsprechen
und ich nichts verschwiegen habe.
Die Strafbarkeit einer falschen eidesstattlichen Versicherung ist mir bekannt,
namentlich die Strafandrohung gemäß § 156 StGB bis zu drei Jahren Freiheitsstrafe
oder Geldstrafe bei vorsätzlicher Begehung der Tat bzw. gemäß \\ § 163
StGB bis zu einem Jahr Freiheitsstrafe oder Geldstrafe bei fahrlässiger Begehung. \\
\\[1.5cm]
Datum:	\hrulefill\enspace Unterschrift: \hrulefill
% Ende ----------------------------------------------------------------------------------------------------------------- % Einbindung der Versicherugn an Eides Statt

	\chapter*{Zusammenfassung} 

\lipsum[1-3]
 % Einbindung der Zusammenfassung
	
	\tableofcontents  % Einbinden des Inhaltsverzeichnisses
	\newpage

	\listoffigures  % Einbinden des Abbildungsverzeichnisses
	\addcontentsline{toc}{chapter}{Abbildungsverzeichnis}
	\newpage
	
	\listoftables  % Einbinden des Tabellenverzeichnisses
	\addcontentsline{toc}{chapter}{Tabellenverzeichnis}
	\newpage

	% \listof{formula}{Formelverzeichnis} % Einbinden des Formelverzeichnisses
	% \newpage

	% Abkürzungsverzeichnis ------------------------------------------------------------------------------------------------
\chapter*{Abkürzungsverzeichnis}
\addcontentsline{toc}{chapter}{Abkürzungsverzeichnis}
\label{sec:abkuerzung}
\begin{acronym}[XXXXXX]
    \acro{A/D}{Analog-Digital}
\end{acronym}
% Ende ----------------------------------------------------------------------------------------------------------------- % Einbinden des Abkürzungsverzeichnisses

	% \mainmatter
	\cleardoubleoddpage
	\pagenumbering{arabic}

	\chapter*{Zusammenfassung} 

\lipsum[1-3]

	
	\chapter{Einleitung}
\label{chap:Einleitung}

\lipsum[1-5]

 % Einbindung der Einleitung
	
	% Literaturverzeichnis
	\printbibliography  % Einbinden des Literaturverzeichnis
	
\end{document}

% Dokumentenende -------------------------------------------------------------------------------------------------------